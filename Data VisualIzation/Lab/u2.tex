% Options for packages loaded elsewhere
\PassOptionsToPackage{unicode}{hyperref}
\PassOptionsToPackage{hyphens}{url}
%
\documentclass[
]{article}
\usepackage{amsmath,amssymb}
\usepackage{lmodern}
\usepackage{iftex}
\ifPDFTeX
  \usepackage[T1]{fontenc}
  \usepackage[utf8]{inputenc}
  \usepackage{textcomp} % provide euro and other symbols
\else % if luatex or xetex
  \usepackage{unicode-math}
  \defaultfontfeatures{Scale=MatchLowercase}
  \defaultfontfeatures[\rmfamily]{Ligatures=TeX,Scale=1}
\fi
% Use upquote if available, for straight quotes in verbatim environments
\IfFileExists{upquote.sty}{\usepackage{upquote}}{}
\IfFileExists{microtype.sty}{% use microtype if available
  \usepackage[]{microtype}
  \UseMicrotypeSet[protrusion]{basicmath} % disable protrusion for tt fonts
}{}
\makeatletter
\@ifundefined{KOMAClassName}{% if non-KOMA class
  \IfFileExists{parskip.sty}{%
    \usepackage{parskip}
  }{% else
    \setlength{\parindent}{0pt}
    \setlength{\parskip}{6pt plus 2pt minus 1pt}}
}{% if KOMA class
  \KOMAoptions{parskip=half}}
\makeatother
\usepackage{xcolor}
\usepackage[margin=1in]{geometry}
\usepackage{color}
\usepackage{fancyvrb}
\newcommand{\VerbBar}{|}
\newcommand{\VERB}{\Verb[commandchars=\\\{\}]}
\DefineVerbatimEnvironment{Highlighting}{Verbatim}{commandchars=\\\{\}}
% Add ',fontsize=\small' for more characters per line
\usepackage{framed}
\definecolor{shadecolor}{RGB}{248,248,248}
\newenvironment{Shaded}{\begin{snugshade}}{\end{snugshade}}
\newcommand{\AlertTok}[1]{\textcolor[rgb]{0.94,0.16,0.16}{#1}}
\newcommand{\AnnotationTok}[1]{\textcolor[rgb]{0.56,0.35,0.01}{\textbf{\textit{#1}}}}
\newcommand{\AttributeTok}[1]{\textcolor[rgb]{0.77,0.63,0.00}{#1}}
\newcommand{\BaseNTok}[1]{\textcolor[rgb]{0.00,0.00,0.81}{#1}}
\newcommand{\BuiltInTok}[1]{#1}
\newcommand{\CharTok}[1]{\textcolor[rgb]{0.31,0.60,0.02}{#1}}
\newcommand{\CommentTok}[1]{\textcolor[rgb]{0.56,0.35,0.01}{\textit{#1}}}
\newcommand{\CommentVarTok}[1]{\textcolor[rgb]{0.56,0.35,0.01}{\textbf{\textit{#1}}}}
\newcommand{\ConstantTok}[1]{\textcolor[rgb]{0.00,0.00,0.00}{#1}}
\newcommand{\ControlFlowTok}[1]{\textcolor[rgb]{0.13,0.29,0.53}{\textbf{#1}}}
\newcommand{\DataTypeTok}[1]{\textcolor[rgb]{0.13,0.29,0.53}{#1}}
\newcommand{\DecValTok}[1]{\textcolor[rgb]{0.00,0.00,0.81}{#1}}
\newcommand{\DocumentationTok}[1]{\textcolor[rgb]{0.56,0.35,0.01}{\textbf{\textit{#1}}}}
\newcommand{\ErrorTok}[1]{\textcolor[rgb]{0.64,0.00,0.00}{\textbf{#1}}}
\newcommand{\ExtensionTok}[1]{#1}
\newcommand{\FloatTok}[1]{\textcolor[rgb]{0.00,0.00,0.81}{#1}}
\newcommand{\FunctionTok}[1]{\textcolor[rgb]{0.00,0.00,0.00}{#1}}
\newcommand{\ImportTok}[1]{#1}
\newcommand{\InformationTok}[1]{\textcolor[rgb]{0.56,0.35,0.01}{\textbf{\textit{#1}}}}
\newcommand{\KeywordTok}[1]{\textcolor[rgb]{0.13,0.29,0.53}{\textbf{#1}}}
\newcommand{\NormalTok}[1]{#1}
\newcommand{\OperatorTok}[1]{\textcolor[rgb]{0.81,0.36,0.00}{\textbf{#1}}}
\newcommand{\OtherTok}[1]{\textcolor[rgb]{0.56,0.35,0.01}{#1}}
\newcommand{\PreprocessorTok}[1]{\textcolor[rgb]{0.56,0.35,0.01}{\textit{#1}}}
\newcommand{\RegionMarkerTok}[1]{#1}
\newcommand{\SpecialCharTok}[1]{\textcolor[rgb]{0.00,0.00,0.00}{#1}}
\newcommand{\SpecialStringTok}[1]{\textcolor[rgb]{0.31,0.60,0.02}{#1}}
\newcommand{\StringTok}[1]{\textcolor[rgb]{0.31,0.60,0.02}{#1}}
\newcommand{\VariableTok}[1]{\textcolor[rgb]{0.00,0.00,0.00}{#1}}
\newcommand{\VerbatimStringTok}[1]{\textcolor[rgb]{0.31,0.60,0.02}{#1}}
\newcommand{\WarningTok}[1]{\textcolor[rgb]{0.56,0.35,0.01}{\textbf{\textit{#1}}}}
\usepackage{graphicx}
\makeatletter
\def\maxwidth{\ifdim\Gin@nat@width>\linewidth\linewidth\else\Gin@nat@width\fi}
\def\maxheight{\ifdim\Gin@nat@height>\textheight\textheight\else\Gin@nat@height\fi}
\makeatother
% Scale images if necessary, so that they will not overflow the page
% margins by default, and it is still possible to overwrite the defaults
% using explicit options in \includegraphics[width, height, ...]{}
\setkeys{Gin}{width=\maxwidth,height=\maxheight,keepaspectratio}
% Set default figure placement to htbp
\makeatletter
\def\fps@figure{htbp}
\makeatother
\setlength{\emergencystretch}{3em} % prevent overfull lines
\providecommand{\tightlist}{%
  \setlength{\itemsep}{0pt}\setlength{\parskip}{0pt}}
\setcounter{secnumdepth}{-\maxdimen} % remove section numbering
\ifLuaTeX
  \usepackage{selnolig}  % disable illegal ligatures
\fi
\IfFileExists{bookmark.sty}{\usepackage{bookmark}}{\usepackage{hyperref}}
\IfFileExists{xurl.sty}{\usepackage{xurl}}{} % add URL line breaks if available
\urlstyle{same} % disable monospaced font for URLs
\hypersetup{
  pdftitle={Data Visualization Lab 7},
  pdfauthor={Aryan Vigyat},
  hidelinks,
  pdfcreator={LaTeX via pandoc}}

\title{Data Visualization Lab 7}
\author{Aryan Vigyat}
\date{2023-02-09}

\begin{document}
\maketitle

Loading the Libraries Neeeded

\begin{Shaded}
\begin{Highlighting}[]
\FunctionTok{library}\NormalTok{(igraph)}
\end{Highlighting}
\end{Shaded}

\begin{verbatim}
## 
## Attaching package: 'igraph'
\end{verbatim}

\begin{verbatim}
## The following objects are masked from 'package:stats':
## 
##     decompose, spectrum
\end{verbatim}

\begin{verbatim}
## The following object is masked from 'package:base':
## 
##     union
\end{verbatim}

\begin{enumerate}
\def\labelenumi{\arabic{enumi}.}
\tightlist
\item
  Loading the Graoh with Adjacency Matrix
\end{enumerate}

\begin{Shaded}
\begin{Highlighting}[]
\NormalTok{dat}\OtherTok{=}\FunctionTok{read.csv}\NormalTok{(}\FunctionTok{file.choose}\NormalTok{(),}\AttributeTok{header=}\ConstantTok{TRUE}\NormalTok{,}\AttributeTok{sep=}\StringTok{\textquotesingle{},\textquotesingle{}}\NormalTok{, }\AttributeTok{row.names=}\DecValTok{1}\NormalTok{,}\AttributeTok{check.names=}\ConstantTok{FALSE}\NormalTok{)}
\NormalTok{m}\OtherTok{=}\FunctionTok{as.matrix}\NormalTok{(dat)}
\NormalTok{net}\OtherTok{=}\FunctionTok{graph.adjacency}\NormalTok{(m,}\AttributeTok{mode=}\StringTok{"directed"}\NormalTok{,}\AttributeTok{weighted=}\ConstantTok{TRUE}\NormalTok{,}\AttributeTok{diag=}\ConstantTok{FALSE}\NormalTok{)}
\FunctionTok{V}\NormalTok{(net)}\SpecialCharTok{$}\NormalTok{color }\OtherTok{\textless{}{-}} \StringTok{"yellow"}
\FunctionTok{V}\NormalTok{(net)}\SpecialCharTok{$}\NormalTok{number }\OtherTok{\textless{}{-}} \FunctionTok{sample}\NormalTok{(}\DecValTok{1}\SpecialCharTok{:}\DecValTok{50}\NormalTok{, }\FunctionTok{vcount}\NormalTok{(net), }\AttributeTok{replace=}\ConstantTok{TRUE}\NormalTok{)}
\FunctionTok{V}\NormalTok{(net)[ number }\SpecialCharTok{\textless{}} \DecValTok{20}\NormalTok{ ]}\SpecialCharTok{$}\NormalTok{color }\OtherTok{\textless{}{-}} \StringTok{"lightblue"}

\FunctionTok{E}\NormalTok{(net)}\SpecialCharTok{$}\NormalTok{weight }\OtherTok{\textless{}{-}} \FunctionTok{runif}\NormalTok{(}\FunctionTok{ecount}\NormalTok{(net))}
\FunctionTok{E}\NormalTok{(net)}\SpecialCharTok{$}\NormalTok{width }\OtherTok{\textless{}{-}} \DecValTok{1}
\FunctionTok{E}\NormalTok{(net)}\SpecialCharTok{$}\NormalTok{color }\OtherTok{\textless{}{-}} \StringTok{"red"}
\FunctionTok{E}\NormalTok{(net)[ weight }\SpecialCharTok{\textless{}} \FloatTok{0.5}\NormalTok{ ]}\SpecialCharTok{$}\NormalTok{width }\OtherTok{\textless{}{-}} \DecValTok{4}
\FunctionTok{E}\NormalTok{(net)[ weight }\SpecialCharTok{\textless{}} \FloatTok{0.5}\NormalTok{ ]}\SpecialCharTok{$}\NormalTok{color }\OtherTok{\textless{}{-}} \StringTok{"brown"}
\FunctionTok{plot}\NormalTok{(net, }\AttributeTok{layout =}\NormalTok{ layout.fruchterman.reingold,}\AttributeTok{vertex.label=}\FunctionTok{V}\NormalTok{(net)}\SpecialCharTok{$}\NormalTok{number)}
\end{Highlighting}
\end{Shaded}

\includegraphics{u2_files/figure-latex/unnamed-chunk-2-1.pdf} 2. Loading
the Graph with EdgeList

\begin{Shaded}
\begin{Highlighting}[]
\NormalTok{e1}\OtherTok{=}\FunctionTok{read.table}\NormalTok{(}\FunctionTok{file.choose}\NormalTok{(),}\AttributeTok{sep=}\StringTok{","}\NormalTok{,}\AttributeTok{header=}\NormalTok{T)}
\NormalTok{g2 }\OtherTok{\textless{}{-}} \FunctionTok{graph.data.frame}\NormalTok{(e1)}
\FunctionTok{V}\NormalTok{(g2)}\SpecialCharTok{$}\NormalTok{number}\OtherTok{=}\FunctionTok{sample}\NormalTok{(}\DecValTok{1}\SpecialCharTok{:}\DecValTok{50}\NormalTok{, }\FunctionTok{vcount}\NormalTok{(g2), }\AttributeTok{replace=}\ConstantTok{TRUE}\NormalTok{)}
\FunctionTok{V}\NormalTok{(g2)}\SpecialCharTok{$}\NormalTok{color}\OtherTok{=}\StringTok{"yellow"}
\FunctionTok{V}\NormalTok{(g2)[ number }\SpecialCharTok{\textless{}} \DecValTok{20}\NormalTok{ ]}\SpecialCharTok{$}\NormalTok{color }\OtherTok{\textless{}{-}} \StringTok{"lightblue"}
\FunctionTok{E}\NormalTok{(g2)}\SpecialCharTok{$}\NormalTok{weight }\OtherTok{\textless{}{-}} \FunctionTok{runif}\NormalTok{(}\FunctionTok{ecount}\NormalTok{(g2))}
\FunctionTok{E}\NormalTok{(g2)}\SpecialCharTok{$}\NormalTok{width }\OtherTok{\textless{}{-}} \DecValTok{1}
\FunctionTok{E}\NormalTok{(g2)}\SpecialCharTok{$}\NormalTok{color }\OtherTok{\textless{}{-}} \StringTok{"red"}
\FunctionTok{E}\NormalTok{(g2)[ weight }\SpecialCharTok{\textless{}} \FloatTok{0.5}\NormalTok{ ]}\SpecialCharTok{$}\NormalTok{width }\OtherTok{\textless{}{-}} \DecValTok{4}
\FunctionTok{E}\NormalTok{(g2)[ weight }\SpecialCharTok{\textless{}} \FloatTok{0.5}\NormalTok{ ]}\SpecialCharTok{$}\NormalTok{color }\OtherTok{\textless{}{-}} \StringTok{"brown"}
\FunctionTok{plot}\NormalTok{(g2, }\AttributeTok{layout =}\NormalTok{ layout.fruchterman.reingold,}\AttributeTok{vertex.label=}\FunctionTok{V}\NormalTok{(g2)}\SpecialCharTok{$}\NormalTok{number)}
\end{Highlighting}
\end{Shaded}

\includegraphics{u2_files/figure-latex/unnamed-chunk-3-1.pdf} 3. Display
the edges \& vertices, the network as matrix and the names of vertices

\begin{Shaded}
\begin{Highlighting}[]
\CommentTok{\#For Graph 1}
\FunctionTok{get.edgelist}\NormalTok{(net)}
\end{Highlighting}
\end{Shaded}

\begin{verbatim}
##       [,1]   [,2]  
##  [1,] "2000" "2001"
##  [2,] "2000" "2003"
##  [3,] "2000" "2005"
##  [4,] "2000" "2007"
##  [5,] "2001" "2000"
##  [6,] "2001" "2002"
##  [7,] "2001" "2004"
##  [8,] "2001" "2006"
##  [9,] "2001" "2007"
## [10,] "2001" "2009"
## [11,] "2001" "2010"
## [12,] "2002" "2000"
## [13,] "2002" "2001"
## [14,] "2002" "2004"
## [15,] "2003" "2000"
## [16,] "2003" "2001"
## [17,] "2003" "2002"
## [18,] "2003" "2005"
## [19,] "2003" "2007"
## [20,] "2003" "2009"
## [21,] "2004" "2000"
## [22,] "2004" "2002"
## [23,] "2004" "2005"
## [24,] "2004" "2007"
## [25,] "2004" "2009"
## [26,] "2004" "2010"
## [27,] "2005" "2002"
## [28,] "2005" "2003"
## [29,] "2005" "2004"
## [30,] "2005" "2006"
## [31,] "2005" "2010"
## [32,] "2006" "2002"
## [33,] "2006" "2004"
## [34,] "2006" "2008"
## [35,] "2006" "2010"
## [36,] "2007" "2002"
## [37,] "2007" "2003"
## [38,] "2007" "2004"
## [39,] "2007" "2006"
## [40,] "2007" "2010"
## [41,] "2008" "2000"
## [42,] "2008" "2004"
## [43,] "2008" "2006"
## [44,] "2008" "2009"
## [45,] "2009" "2000"
## [46,] "2009" "2001"
## [47,] "2009" "2002"
## [48,] "2009" "2004"
## [49,] "2009" "2005"
## [50,] "2009" "2007"
## [51,] "2009" "2008"
## [52,] "2010" "2001"
## [53,] "2010" "2002"
## [54,] "2010" "2003"
## [55,] "2010" "2005"
## [56,] "2010" "2006"
## [57,] "2010" "2007"
## [58,] "2010" "2008"
## [59,] "2010" "2009"
\end{verbatim}

\begin{Shaded}
\begin{Highlighting}[]
\FunctionTok{get.adjacency}\NormalTok{(net)}
\end{Highlighting}
\end{Shaded}

\begin{verbatim}
## 11 x 11 sparse Matrix of class "dgCMatrix"
\end{verbatim}

\begin{verbatim}
##   [[ suppressing 11 column names '2000', '2001', '2002' ... ]]
\end{verbatim}

\begin{verbatim}
##                           
## 2000 . 1 . 1 . 1 . 1 . . .
## 2001 1 . 1 . 1 . 1 1 . 1 1
## 2002 1 1 . . 1 . . . . . .
## 2003 1 1 1 . . 1 . 1 . 1 .
## 2004 1 . 1 . . 1 . 1 . 1 1
## 2005 . . 1 1 1 . 1 . . . 1
## 2006 . . 1 . 1 . . . 1 . 1
## 2007 . . 1 1 1 . 1 . . . 1
## 2008 1 . . . 1 . 1 . . 1 .
## 2009 1 1 1 . 1 1 . 1 1 . .
## 2010 . 1 1 1 . 1 1 1 1 1 .
\end{verbatim}

\begin{Shaded}
\begin{Highlighting}[]
\CommentTok{\#For Graph 2}
\FunctionTok{get.edgelist}\NormalTok{(g2)}
\end{Highlighting}
\end{Shaded}

\begin{verbatim}
##       [,1]   [,2]  
##  [1,] "2000" "2001"
##  [2,] "2001" "2002"
##  [3,] "2002" "2004"
##  [4,] "2003" "2004"
##  [5,] "2004" "2003"
##  [6,] "2004" "2002"
##  [7,] "2006" "2008"
##  [8,] "2006" "2007"
##  [9,] "2008" "2010"
## [10,] "2008" "2000"
## [11,] "2010" "2000"
\end{verbatim}

\begin{Shaded}
\begin{Highlighting}[]
\FunctionTok{get.adjacency}\NormalTok{(g2)}
\end{Highlighting}
\end{Shaded}

\begin{verbatim}
## 9 x 9 sparse Matrix of class "dgCMatrix"
##      2000 2001 2002 2003 2004 2006 2008 2010 2007
## 2000    .    1    .    .    .    .    .    .    .
## 2001    .    .    1    .    .    .    .    .    .
## 2002    .    .    .    .    1    .    .    .    .
## 2003    .    .    .    .    1    .    .    .    .
## 2004    .    .    1    1    .    .    .    .    .
## 2006    .    .    .    .    .    .    1    .    1
## 2008    1    .    .    .    .    .    .    1    .
## 2010    1    .    .    .    .    .    .    .    .
## 2007    .    .    .    .    .    .    .    .    .
\end{verbatim}

\begin{enumerate}
\def\labelenumi{\arabic{enumi}.}
\setcounter{enumi}{3}
\tightlist
\item
  Find the count of vertices and edges of the created graph
\end{enumerate}

\begin{Shaded}
\begin{Highlighting}[]
\CommentTok{\#For Graph 1}
\FunctionTok{print}\NormalTok{(}\FunctionTok{vcount}\NormalTok{(net))}
\end{Highlighting}
\end{Shaded}

\begin{verbatim}
## [1] 11
\end{verbatim}

\begin{Shaded}
\begin{Highlighting}[]
\FunctionTok{print}\NormalTok{(}\FunctionTok{ecount}\NormalTok{(net))}
\end{Highlighting}
\end{Shaded}

\begin{verbatim}
## [1] 59
\end{verbatim}

\begin{Shaded}
\begin{Highlighting}[]
\CommentTok{\#For Graph 2}
\FunctionTok{print}\NormalTok{(}\FunctionTok{vcount}\NormalTok{(g2))}
\end{Highlighting}
\end{Shaded}

\begin{verbatim}
## [1] 9
\end{verbatim}

\begin{Shaded}
\begin{Highlighting}[]
\FunctionTok{print}\NormalTok{(}\FunctionTok{ecount}\NormalTok{(g2))}
\end{Highlighting}
\end{Shaded}

\begin{verbatim}
## [1] 11
\end{verbatim}

\begin{enumerate}
\def\labelenumi{\arabic{enumi}.}
\setcounter{enumi}{4}
\tightlist
\item
  Display the adjacency vertices of each vertex(individual) in the
  created graph
\end{enumerate}

\begin{Shaded}
\begin{Highlighting}[]
\CommentTok{\#For Graph 1}
\ControlFlowTok{for}\NormalTok{ (i }\ControlFlowTok{in} \DecValTok{1}\SpecialCharTok{:}\FunctionTok{vcount}\NormalTok{(net)) \{}
  \CommentTok{\# get the adjacent vertices of each vertex}
\NormalTok{  adj\_vertices }\OtherTok{\textless{}{-}} \FunctionTok{neighbors}\NormalTok{(net, i)}
  
  \CommentTok{\# print the vertex and its adjacent vertices}
  \FunctionTok{cat}\NormalTok{(}\StringTok{"Vertex"}\NormalTok{, i, }\StringTok{"has the following neighbouring vertices:"}\NormalTok{, adj\_vertices, }\StringTok{"}\SpecialCharTok{\textbackslash{}n}\StringTok{"}\NormalTok{)}
\NormalTok{\}}
\end{Highlighting}
\end{Shaded}

\begin{verbatim}
## Vertex 1 has the following neighbouring vertices: 2 4 6 8 
## Vertex 2 has the following neighbouring vertices: 1 3 5 7 8 10 11 
## Vertex 3 has the following neighbouring vertices: 1 2 5 
## Vertex 4 has the following neighbouring vertices: 1 2 3 6 8 10 
## Vertex 5 has the following neighbouring vertices: 1 3 6 8 10 11 
## Vertex 6 has the following neighbouring vertices: 3 4 5 7 11 
## Vertex 7 has the following neighbouring vertices: 3 5 9 11 
## Vertex 8 has the following neighbouring vertices: 3 4 5 7 11 
## Vertex 9 has the following neighbouring vertices: 1 5 7 10 
## Vertex 10 has the following neighbouring vertices: 1 2 3 5 6 8 9 
## Vertex 11 has the following neighbouring vertices: 2 3 4 6 7 8 9 10
\end{verbatim}

\begin{Shaded}
\begin{Highlighting}[]
\CommentTok{\#For Graph 2}
\ControlFlowTok{for}\NormalTok{ (i }\ControlFlowTok{in} \DecValTok{1}\SpecialCharTok{:}\FunctionTok{vcount}\NormalTok{(g2)) \{}
  \CommentTok{\# get the adjacent vertices of each vertex}
\NormalTok{  adj\_vertices }\OtherTok{\textless{}{-}} \FunctionTok{neighbors}\NormalTok{(g2, i)}
  
  \CommentTok{\# print the vertex and its adjacent vertices}
  \FunctionTok{cat}\NormalTok{(}\StringTok{"Vertex"}\NormalTok{, i, }\StringTok{"has the following neighbouring vertices:"}\NormalTok{, adj\_vertices, }\StringTok{"}\SpecialCharTok{\textbackslash{}n}\StringTok{"}\NormalTok{)}
\NormalTok{\}}
\end{Highlighting}
\end{Shaded}

\begin{verbatim}
## Vertex 1 has the following neighbouring vertices: 2 
## Vertex 2 has the following neighbouring vertices: 3 
## Vertex 3 has the following neighbouring vertices: 5 
## Vertex 4 has the following neighbouring vertices: 5 
## Vertex 5 has the following neighbouring vertices: 3 4 
## Vertex 6 has the following neighbouring vertices: 7 9 
## Vertex 7 has the following neighbouring vertices: 1 8 
## Vertex 8 has the following neighbouring vertices: 1 
## Vertex 9 has the following neighbouring vertices:
\end{verbatim}

6.Find the min and max degree of the created graph

\begin{Shaded}
\begin{Highlighting}[]
\CommentTok{\#For Graph 1}
\NormalTok{vertex\_degrees }\OtherTok{\textless{}{-}} \FunctionTok{degree}\NormalTok{(net)}

\CommentTok{\# find the minimum and maximum degree}
\NormalTok{min\_degree }\OtherTok{\textless{}{-}} \FunctionTok{min}\NormalTok{(vertex\_degrees)}
\NormalTok{max\_degree }\OtherTok{\textless{}{-}} \FunctionTok{max}\NormalTok{(vertex\_degrees)}

\CommentTok{\# print the minimum and maximum degree}
\FunctionTok{cat}\NormalTok{(}\StringTok{"The minimum degree is"}\NormalTok{, min\_degree, }\StringTok{"}\SpecialCharTok{\textbackslash{}n}\StringTok{"}\NormalTok{)}
\end{Highlighting}
\end{Shaded}

\begin{verbatim}
## The minimum degree is 7
\end{verbatim}

\begin{Shaded}
\begin{Highlighting}[]
\FunctionTok{cat}\NormalTok{(}\StringTok{"The maximum degree is"}\NormalTok{, max\_degree, }\StringTok{"}\SpecialCharTok{\textbackslash{}n}\StringTok{"}\NormalTok{)}
\end{Highlighting}
\end{Shaded}

\begin{verbatim}
## The maximum degree is 13
\end{verbatim}

\begin{Shaded}
\begin{Highlighting}[]
\CommentTok{\#For Graph 2}
\NormalTok{vertex\_degrees }\OtherTok{\textless{}{-}} \FunctionTok{degree}\NormalTok{(g2)}

\CommentTok{\# find the minimum and maximum degree}
\NormalTok{min\_degree }\OtherTok{\textless{}{-}} \FunctionTok{min}\NormalTok{(vertex\_degrees)}
\NormalTok{max\_degree }\OtherTok{\textless{}{-}} \FunctionTok{max}\NormalTok{(vertex\_degrees)}

\CommentTok{\# print the minimum and maximum degree}
\FunctionTok{cat}\NormalTok{(}\StringTok{"The minimum degree is"}\NormalTok{, min\_degree, }\StringTok{"}\SpecialCharTok{\textbackslash{}n}\StringTok{"}\NormalTok{)}
\end{Highlighting}
\end{Shaded}

\begin{verbatim}
## The minimum degree is 1
\end{verbatim}

\begin{Shaded}
\begin{Highlighting}[]
\FunctionTok{cat}\NormalTok{(}\StringTok{"The maximum degree is"}\NormalTok{, max\_degree, }\StringTok{"}\SpecialCharTok{\textbackslash{}n}\StringTok{"}\NormalTok{)}
\end{Highlighting}
\end{Shaded}

\begin{verbatim}
## The maximum degree is 4
\end{verbatim}

7. Create \& set vertex attribute property named profit and
values(``+'', ``-'', ``+'', ``-'', ``+'', ``-'', ``+'', ``-'', ``+'')

\begin{Shaded}
\begin{Highlighting}[]
\NormalTok{vertex\_attributes }\OtherTok{\textless{}{-}} \FunctionTok{c}\NormalTok{(}\StringTok{"+"}\NormalTok{, }\StringTok{"{-}"}\NormalTok{, }\StringTok{"+"}\NormalTok{, }\StringTok{"{-}"}\NormalTok{, }\StringTok{"+"}\NormalTok{, }\StringTok{"{-}"}\NormalTok{, }\StringTok{"+"}\NormalTok{, }\StringTok{"{-}"}\NormalTok{, }\StringTok{"+"}\NormalTok{)}

\CommentTok{\# set the vertex attributes as the "profit" property}
\FunctionTok{set\_vertex\_attr}\NormalTok{(g2, }\StringTok{"profit"}\NormalTok{, }\AttributeTok{value =}\NormalTok{ vertex\_attributes)}
\end{Highlighting}
\end{Shaded}

\begin{verbatim}
## IGRAPH b875c5a DNW- 9 11 -- 
## + attr: name (v/c), number (v/n), color (v/c), profit (v/c), weight
## | (e/n), width (e/n), color (e/c)
## + edges from b875c5a (vertex names):
##  [1] 2000->2001 2001->2002 2002->2004 2003->2004 2004->2003 2004->2002
##  [7] 2006->2008 2006->2007 2008->2010 2008->2000 2010->2000
\end{verbatim}

\begin{Shaded}
\begin{Highlighting}[]
\CommentTok{\# check the vertex attributes}
\FunctionTok{summary}\NormalTok{(g2)}
\end{Highlighting}
\end{Shaded}

\begin{verbatim}
## IGRAPH b875c5a DNW- 9 11 -- 
## + attr: name (v/c), number (v/n), color (v/c), weight (e/n), width
## | (e/n), color (e/c)
\end{verbatim}

8. Create \& set vertex attribute property named type and values(either
leap or non-leap year)

\begin{Shaded}
\begin{Highlighting}[]
\FunctionTok{V}\NormalTok{(g2)}\SpecialCharTok{$}\NormalTok{name }\OtherTok{\textless{}{-}} \FunctionTok{as.numeric}\NormalTok{(}\FunctionTok{V}\NormalTok{(g2)}\SpecialCharTok{$}\NormalTok{name)}
\FunctionTok{V}\NormalTok{(g2)}\SpecialCharTok{$}\NormalTok{type }\OtherTok{=} \FunctionTok{ifelse}\NormalTok{(}\FunctionTok{V}\NormalTok{(g2)}\SpecialCharTok{$}\NormalTok{name }\SpecialCharTok{\%\%} \DecValTok{4} \SpecialCharTok{==} \DecValTok{0}\NormalTok{, }\StringTok{"Leap"}\NormalTok{, }\StringTok{"Non{-}Leap"}\NormalTok{)}
\FunctionTok{V}\NormalTok{(g2)}\SpecialCharTok{$}\NormalTok{type}
\end{Highlighting}
\end{Shaded}

\begin{verbatim}
## [1] "Leap"     "Non-Leap" "Non-Leap" "Non-Leap" "Leap"     "Non-Leap" "Leap"    
## [8] "Non-Leap" "Non-Leap"
\end{verbatim}

9.Create \& set edge attribute named weight and values (if edge exits in
between leap year vertices then 5 else 1 )

\begin{Shaded}
\begin{Highlighting}[]
\FunctionTok{E}\NormalTok{(g2)}\SpecialCharTok{$}\NormalTok{weight }\OtherTok{\textless{}{-}} \FunctionTok{as.numeric}\NormalTok{(}\FunctionTok{ifelse}\NormalTok{(}\FunctionTok{all}\NormalTok{(}\FunctionTok{V}\NormalTok{(g2)}\SpecialCharTok{$}\NormalTok{type }\SpecialCharTok{==} \StringTok{"Leap"}\NormalTok{), }\DecValTok{5}\NormalTok{, }\DecValTok{1}\NormalTok{))}
\FunctionTok{print}\NormalTok{(}\FunctionTok{E}\NormalTok{(g2)}\SpecialCharTok{$}\NormalTok{weight)}
\end{Highlighting}
\end{Shaded}

\begin{verbatim}
##  [1] 1 1 1 1 1 1 1 1 1 1 1
\end{verbatim}

10. Convert the created un-directed graph into directed graph based on
the following rule a. edge directed towards high value vertex

\begin{Shaded}
\begin{Highlighting}[]
\NormalTok{dg }\OtherTok{=} \FunctionTok{as.directed}\NormalTok{(g2, }\AttributeTok{mode =} \StringTok{"arbitrary"}\NormalTok{)}
\FunctionTok{plot}\NormalTok{(dg,}\AttributeTok{layout=}\NormalTok{layout.circle)}
\end{Highlighting}
\end{Shaded}

\includegraphics{u2_files/figure-latex/unnamed-chunk-11-1.pdf} 12.
Display the adjacency matrix of the resultant directed graph

\begin{Shaded}
\begin{Highlighting}[]
\FunctionTok{E}\NormalTok{(dg, }\AttributeTok{P=}\ConstantTok{NULL}\NormalTok{, }\AttributeTok{path=}\ConstantTok{NULL}\NormalTok{, }\AttributeTok{directed=}\ConstantTok{TRUE}\NormalTok{)}
\end{Highlighting}
\end{Shaded}

\begin{verbatim}
## + 11/11 edges from b91c4b5 (vertex names):
##  [1] 2000->2001 2001->2002 2002->2004 2003->2004 2004->2003 2004->2002
##  [7] 2006->2008 2006->2007 2008->2010 2008->2000 2010->2000
\end{verbatim}

\begin{enumerate}
\def\labelenumi{\arabic{enumi}.}
\setcounter{enumi}{12}
\tightlist
\item
  Display the in-degree and out-degree of each vertex of resultant
  directed graph
\end{enumerate}

\begin{Shaded}
\begin{Highlighting}[]
\NormalTok{indegree }\OtherTok{\textless{}{-}} \FunctionTok{degree}\NormalTok{(dg, }\AttributeTok{mode =} \StringTok{"in"}\NormalTok{)}
\NormalTok{outdegree }\OtherTok{\textless{}{-}} \FunctionTok{degree}\NormalTok{(dg, }\AttributeTok{mode =} \StringTok{"out"}\NormalTok{)}

\CommentTok{\# Combine the indegree and outdegree into a data frame}
\NormalTok{df }\OtherTok{\textless{}{-}} \FunctionTok{data.frame}\NormalTok{(}\AttributeTok{vertex =} \FunctionTok{V}\NormalTok{(dg)}\SpecialCharTok{$}\NormalTok{name, }\AttributeTok{indegree =}\NormalTok{ indegree, }\AttributeTok{outdegree =}\NormalTok{ outdegree)}

\CommentTok{\# Display the result}
\FunctionTok{print}\NormalTok{(df)}
\end{Highlighting}
\end{Shaded}

\begin{verbatim}
##      vertex indegree outdegree
## 2000   2000        2         1
## 2001   2001        1         1
## 2002   2002        2         1
## 2003   2003        1         1
## 2004   2004        2         2
## 2006   2006        0         2
## 2008   2008        1         2
## 2010   2010        1         1
## 2007   2007        1         0
\end{verbatim}

\end{document}
